%!TEX program = xelatex
\documentclass[dvipsnames, svgnames,a4paper,11pt]{article}
% ----------------------------------------------------
%   中山大学物理与天文学院本科实验报告模板
%   作者:Huanyu Shi,2019级
%   知乎:https://www.zhihu.com/people/za-ran-zhu-fu-liu-xing
%   Github:https://github.com/huanyushi/SYSU-SPA-Labreport-Template
%   Last update : 2023.4.10
% ----------------------------------------------------

\input{Settings} % 导入模板的相关设置
\usepackage{lipsum}
\usepackage{enumitem}
\usepackage{tabularray}  %绘制表格时可以更加方便添加框线
\setlist[enumerate]{label=\textup{(\arabic*)}}



%---------------------------------------------------------------------
%	正文
%---------------------------------------------------------------------

\begin{document}


\begin{table}
	\renewcommand\arraystretch{1.7}
	\begin{tabularx}{\textwidth}{
		|X|X|X|X
		|X|X|X|X|}
	\hline
	\multicolumn{2}{|c|}{预习报告}&\multicolumn{2}{|c|}{实验记录}&\multicolumn{2}{|c|}{分析讨论}&\multicolumn{2}{|c|}{总成绩}\\
	\hline
	\LARGE25 & & \LARGE30 & & \LARGE25 & & \LARGE80 & \\
	\hline
	\end{tabularx}
\end{table}


\begin{table}
	\renewcommand\arraystretch{1.7}
	\begin{tabularx}{\textwidth}{|X|X|X|X|}
	\hline
	专业:& 物理学 &年级:& 2022级\\
	\hline
	姓名:& 戴鹏辉  & 学号: & 2344016 \\
	\hline
	日期:& 2024/9/23 & 教师签名:& \\
	\hline
	\end{tabularx}
\end{table}

\begin{center}
	\LARGE D3 \quad 量子密钥分发
\end{center}

\textbf{【实验报告注意事项】}
\begin{enumerate}
	\item 实验报告由三部分组成:
	\begin{enumerate}
		\item 预习报告:(提前一周)认真研读\underline{\textbf{实验讲义}},弄清实验原理;实验所需的仪器设备、用具及其使用(强烈建议到实验室预习),完成课前预习思考题;了解实验需要测量的物理量,并根据要求提前准备实验记录表格(第一循环实验已由教师提供模板,可以打印)。预习成绩低于10分(共20分)者不能做实验。
	    \item 实验记录:认真、客观记录实验条件、实验过程中的现象以及数据。实验记录请用珠笔或者钢笔书写并签名(\textcolor{red}{\textbf{用铅笔记录的被认为无效}})。\textcolor{red}{\textbf{保持原始记录,包括写错删除部分,如因误记需要修改记录,必须按规范修改。}}(不得输入电脑打印,但可扫描手记后打印扫描件);离开前请实验教师检查记录并签名。
	    \item 分析讨论:处理实验原始数据(学习仪器使用类型的实验除外),对数据的可靠性和合理性进行分析;按规范呈现数据和结果(图、表),包括数据、图表按顺序编号及其引用;分析物理现象(含回答实验思考题,写出问题思考过程,必要时按规范引用数据);最后得出结论。
	\end{enumerate}
	\textbf{实验报告就是将预习报告、实验记录、和数据处理与分析合起来,加上本页封面。}
	% \item 每次完成实验后的一周内交\textbf{实验报告}(特殊情况不能超过两周)。
	\item 实验报告注意事项
		\begin{enumerate}[label=\roman*.]
			\item 系统工作温度在 15°~ 30°的环境中,尤其避免过高温度下使用本系统。
			\item 实验元件会单独给出,实验前检查是否完整。除给出的元件外,整体密钥分发系统不要触碰。
			\item 镜筒等光机械安装时,螺丝拧紧避免晃动。光机械元件的调节旋钮,安装前,将螺丝行程旋至中间位置,方便实验过程中调节。
			\item 所有镜片避免用手接触光学面,拿捏过程中,光学面垂直于平台,避免灰尘,使用完收入对应的盒子中。安装镜片需靠近台面,避免镜片跌落摔碎。
			\item 请不要打开单光子探测器的黑色遮盖物。
			\item 不要使眼睛与光路处于同一水平面,不要用手直接接触激光,激光为 30mw 紫外激光,必须戴好护目镜。
		\end{enumerate}
\end{enumerate}


\clearpage
\tableofcontents
\clearpage

\setcounter{section}{0}
\section{D3 \quad 量子密钥分发 \quad\heiti 预习报告}
	
\subsection{实验目的}
\begin{enumerate}
	\item 掌握控制和测量光子的偏振;
	\item 掌握单光子的标定;
	\item 掌握单光子的探测及相应探测器效率的测量;
	\item 掌握 BB84 量子密钥分发过程的数据处理。
	
\end{enumerate}

\subsection{仪器用具}
\begin{table}[htbp]
	\centering
	\renewcommand\arraystretch{1.6}
	% \setlength{\tabcolsep}{10mm}
	\caption{偏振测量实验}
	\begin{tabular}{p{0.05\textwidth}|p{0.20\textwidth}|p{0.05\textwidth}|p{0.5\textwidth}}
		\hline
		编号& 仪器用具名称 & 数量 &  主要参数(型号,测量范围,测量精度等) \\
		\hline
		1	&	准直激光器 	& 1 & 波长:404nm,最大功率:150mW\\

		2	&	偏振分光棱镜 	& 2 & 波长:404nm,消光比>500	 \\
		
		3	&	半波片 & 2 &	波长:404nm,零级 	\\
		
		4	&	小型磁性底座	&  & MB105	\\
		
		5	&	PH 系列杆架	& 6 & PH102	\\

		6	&	SP 系列接杆	& 6 & SP104 \\

		7	&	激光器镜架	& 6 & OM311	\\

		8	&	精密棱镜台	& 2 & PPM101	\\

		9	&	偏光镜架	& 2 & PM101	\\

		10	&	可见光功率计& 2 & PM100、S120VC	\\

		11	&	直流稳压电源& 1 & GPD-3303D	\\
		\hline
	\end{tabular}
\end{table}

\begin{table}[htbp]
	\centering
	\renewcommand\arraystretch{1.6}
	% \setlength{\tabcolsep}{10mm}
	\caption{单光子标定的用具}
	\begin{tabular}{p{0.05\textwidth}|p{0.20\textwidth}|p{0.05\textwidth}|p{0.5\textwidth}}
		\hline
		编号& 仪器用具名称 & 数量 &  主要参数(型号,测量范围,测量精度等) \\
		\hline
		1	&	密钥分发系统 	& 1 & 波长:404nm	\\

		2	&	可见光功率计 	& 1 & PM100、S120VC	 \\
		\hline
	\end{tabular}
\end{table}

\begin{table}[htbp]
	\centering
	\renewcommand\arraystretch{1.6}
	% \setlength{\tabcolsep}{10mm}
	\caption{单光子的探测及相应探测器探测效率测量的用具}
	\begin{tabular}{p{0.05\textwidth}|p{0.20\textwidth}|p{0.05\textwidth}|p{0.5\textwidth}}
		\hline
		编号& 仪器用具名称 & 数量 &  主要参数(型号,测量范围,测量精度等) \\
		\hline
		1	&	反射镜 	& 1 & 波长:404nm,45 度入射	\\

		2	&	滤波片 	& 1 & 波长:405nm,带宽:3nm	 \\
		
		3	&	光纤准直器 & 1 &	F671FC-405 	\\
		
		4	&	反射镜折叠架	& 1 & OM402	\\
		
		5	&	透镜固定架	& 1 & LH102	\\

		6	&	光纤耦合架	& 1 & PFC201 \\

		7	&	小型磁性底座	& 3 & MB105	\\

		8	&	PH 系列杆架	& 3 & PH102	\\

		9	&	SP 系列接杆	& 1 & SP104	\\

		10	&	SP 系列接杆	& 2	& SP134	\\

		11	&	可见光功率计	& 1 & PM100、S120VC	\\

		12	&	直流稳压电源& 1 & GPD-3303D	\\
		\hline
	\end{tabular}
\end{table}


\begin{table}[htbp]
	\centering
	\renewcommand\arraystretch{1.6}
	% \setlength{\tabcolsep}{10mm}
	\caption{密钥分发过程数据处理的用具}
	\begin{tabular}{p{0.05\textwidth}|p{0.20\textwidth}|p{0.05\textwidth}|p{0.5\textwidth}}
		\hline
		编号& 仪器用具名称 & 数量 &  主要参数(型号,测量范围,测量精度等) \\
		\hline
		1	&	密钥分发系统 	& 1 & 波长:404nm	\\
		\hline
	\end{tabular}
\end{table}


\subsection{原理概述}


	




\subsection{实验前思考题}

% 思考题1
\begin{question}
	回顾偏振光实验,说明$\lambda / 2$波片,$\lambda / 4$波片的工作原理;
\end{question}


% 思考题2
\begin{question}
	如何检测一个任意方向的线偏振光?
\end{question}



% 思考题3
\begin{question}
	单光子为什么不能直接用普通功率计测量?
\end{question}




% 思考题4
\begin{question}
	检验单光子探测器的探测效率可以用强光吗?
\end{question}






% 思考题5
\begin{question}
	BB84 协议的原理和步骤。
\end{question}





% 思考题6
\begin{question}
	密钥分发过程中,为什么需要有同步信号?
\end{question}




\clearpage
\begin{table}
	\renewcommand\arraystretch{1.7}
	\centering
	\begin{tabularx}{\textwidth}{|X|X|X|X|}
	\hline
	专业:& 物理学 &年级:& 2022级 \\
	\hline
	姓名:& 戴鹏辉 & 学号:& 22344016 \\
	\hline
	室温:& 26℃ & 实验地点: & A509 \\
	\hline
	学生签名:& & 评分: &\\
	\hline
	实验时间:& 2024/5/9 & 教师签名:&\\
	\hline
	\end{tabularx}
\end{table}

\section{D1 \quad 锁相放大器与弱信号测量 \quad\heiti 实验记录}
\subsection{实验内容和步骤}

	\subsubsection{实验一 \quad 选择合适电流量程,设置氩管工作电压}
	
		



\subsection{实验数据记录}

	% 见\cref{fig:data}

	% \begin{figure}[htbp]
	% 	\centering
	% 	\subfloat[原始数据1]
	% 	{\includegraphics[width=0.35\textwidth]{OriginalData1.jpg}\label{fig:data1}}
	% 	\quad
	% 	\subfloat[原始数据2]
	% 	{\includegraphics[width=0.35\textwidth]{OriginalData2.jpg}\label{fig:data2}}
	% 	\quad


	% 	\caption{原始数据}
	% 	\label{fig:data}
	% \end{figure}



%\subsection{原始数据记录}



\subsection{实验过程中遇到的问题记录}

\begin{enumerate}
	\item 
	
\end{enumerate}
	

\clearpage
\begin{table}
	\renewcommand\arraystretch{1.7}
	\begin{tabularx}{\textwidth}{|X|X|X|X|}
	\hline
	专业:& 物理学 &年级:& 2022级\\
	\hline
	姓名: & 戴鹏辉 & 学号:& 22344016\\
	\hline
    日期:& 2024/5/12 & 评分: &\\
	\hline
	\end{tabularx}
\end{table}

\section{D1 \quad 锁相放大器与弱信号测量 \quad\heiti 分析与讨论}

\subsection{实验数据分析}


	\subsubsection{实验一 \quad 选择合适电流量程,设置氩管工作电压}
		
		





\end{document}
