%!TEX program = xelatex
\documentclass[dvipsnames, svgnames,a4paper,11pt]{article}
% ----------------------------------------------------
%   中山大学物理与天文学院本科实验报告模板
%   作者:Huanyu Shi,2019级
%   知乎:https://www.zhihu.com/people/za-ran-zhu-fu-liu-xing
%   Github:https://github.com/huanyushi/SYSU-SPA-Labreport-Template
%   Last update : 2023.4.10
% ----------------------------------------------------

\input{Settings} % 导入模板的相关设置
\usepackage{lipsum}
\usepackage{enumitem}
\usepackage{tabularray}  %绘制表格时可以更加方便添加框线
\setlist[enumerate]{label=\textup{(\arabic*)}}



%---------------------------------------------------------------------
%	正文
%---------------------------------------------------------------------

\begin{document}


\begin{table}
	\renewcommand\arraystretch{1.7}
	\begin{tabularx}{\textwidth}{
		|X|X|X|X
		|X|X|X|X|}
	\hline
	\multicolumn{2}{|c|}{预习报告}&\multicolumn{2}{|c|}{实验记录}&\multicolumn{2}{|c|}{分析讨论}&\multicolumn{2}{|c|}{总成绩}\\
	\hline
	\LARGE25 & & \LARGE30 & & \LARGE25 & & \LARGE80 & \\
	\hline
	\end{tabularx}
\end{table}


\begin{table}
	\renewcommand\arraystretch{1.7}
	\begin{tabularx}{\textwidth}{|X|X|X|X|}
	\hline
	专业:& 物理学 &年级:& 2022级\\
	\hline
	姓名:& 戴鹏辉  & 学号: & 2344016 \\
	\hline
	日期:& 2024/9/20 & 教师签名:& \\
	\hline
	\end{tabularx}
\end{table}

\begin{center}
	\LARGE D1 \quad 锁相放大器与弱信号测量
\end{center}

\textbf{【实验报告注意事项】}
\begin{enumerate}
	\item 实验报告由三部分组成:
	\begin{enumerate}
		\item 预习报告:(提前一周)认真研读\underline{\textbf{实验讲义}},弄清实验原理;实验所需的仪器设备、用具及其使用(强烈建议到实验室预习),完成课前预习思考题;了解实验需要测量的物理量,并根据要求提前准备实验记录表格(第一循环实验已由教师提供模板,可以打印)。预习成绩低于10分(共20分)者不能做实验。
	    \item 实验记录:认真、客观记录实验条件、实验过程中的现象以及数据。实验记录请用珠笔或者钢笔书写并签名(\textcolor{red}{\textbf{用铅笔记录的被认为无效}})。\textcolor{red}{\textbf{保持原始记录,包括写错删除部分,如因误记需要修改记录,必须按规范修改。}}(不得输入电脑打印,但可扫描手记后打印扫描件);离开前请实验教师检查记录并签名。
	    \item 分析讨论:处理实验原始数据(学习仪器使用类型的实验除外),对数据的可靠性和合理性进行分析;按规范呈现数据和结果(图、表),包括数据、图表按顺序编号及其引用;分析物理现象(含回答实验思考题,写出问题思考过程,必要时按规范引用数据);最后得出结论。
	\end{enumerate}
	\textbf{实验报告就是将预习报告、实验记录、和数据处理与分析合起来,加上本页封面。}
	\item 每次完成实验后的一周内交\textbf{实验报告}(特殊情况不能超过两周)。
	\item 实验报告注意事项
		\begin{enumerate}[label=\roman*.]
			\item 
		\end{enumerate}
\end{enumerate}


\clearpage
\tableofcontents
\clearpage

\setcounter{section}{0}
\section{D1 \quad 锁相放大器与弱信号测量——【实验1】锁相放大器工作原理、基本参数与基本操作 \quad\heiti 预习报告}
	
\subsection{实验目的}
\begin{enumerate}
	\item 了解锁相放大器工作原理和特点, 理解信号、 噪声、 信噪比等概念。
	\item 掌握锁相放大器基本参数含义及锁相放大器的基本操作, 学会合理选择或调节参数( 频率、 相位、 灵敏度、 时间常数、 陡降) ; 复习示波器的使用;
	\item 掌握用锁相放大器检测弱信号方法, 通过与示波器比较其检测能力了解其技术优势。
	
\end{enumerate}

\subsection{仪器用具}
\begin{table}[htbp]
	\centering
	\renewcommand\arraystretch{1.6}
	% \setlength{\tabcolsep}{10mm}
	\begin{tabular}{p{0.05\textwidth}|p{0.20\textwidth}|p{0.05\textwidth}|p{0.5\textwidth}}
	\hline
	编号& 仪器用具名称 & 数量 &  主要参数(型号,测量范围,测量精度等) \\
	\hline
	1	&	锁相放大器 	&1 	& OE1022( 系列)\\

	2	&	配套教学实验箱 	&1 	& 		 \\
	
	3	&	示波器 & 1 &	RIGOL DS2202A 	\\
	
	4	&	信号发生器	&1 & RIGOL DG4162	\\
	
	5	&	BNC-BNC 信号线	&	若干 & \\
	\hline
\end{tabular}
\end{table}

\subsection{原理概述}


	




\subsection{实验前思考题}

% 思考题1
\begin{question}
	是否只要与原子发生碰撞的电子能量达到$E_n$-$E_m$,原子就会发生能级跃迁?
\end{question}

	









\clearpage
\begin{table}
	\renewcommand\arraystretch{1.7}
	\centering
	\begin{tabularx}{\textwidth}{|X|X|X|X|}
	\hline
	专业:& 物理学 &年级:& 2022级 \\
	\hline
	姓名:& 戴鹏辉 & 学号:& 22344016 \\
	\hline
	室温:& 26℃ & 实验地点: & A509 \\
	\hline
	学生签名:& & 评分: &\\
	\hline
	实验时间:& 2024/5/9 & 教师签名:&\\
	\hline
	\end{tabularx}
\end{table}

\section{D1 \quad 锁相放大器与弱信号测量 \quad\heiti 实验记录}
\subsection{实验内容和步骤}

	\subsubsection{实验一 \quad 选择合适电流量程,设置氩管工作电压}
	
		



\subsection{实验数据记录}

	% 见\cref{fig:data}

	% \begin{figure}[htbp]
	% 	\centering
	% 	\subfloat[原始数据1]
	% 	{\includegraphics[width=0.35\textwidth]{OriginalData1.jpg}\label{fig:data1}}
	% 	\quad
	% 	\subfloat[原始数据2]
	% 	{\includegraphics[width=0.35\textwidth]{OriginalData2.jpg}\label{fig:data2}}
	% 	\quad


	% 	\caption{原始数据}
	% 	\label{fig:data}
	% \end{figure}



%\subsection{原始数据记录}



\subsection{实验过程中遇到的问题记录}

\begin{enumerate}
	\item 
	
\end{enumerate}
	

\clearpage
\begin{table}
	\renewcommand\arraystretch{1.7}
	\begin{tabularx}{\textwidth}{|X|X|X|X|}
	\hline
	专业:& 物理学 &年级:& 2022级\\
	\hline
	姓名: & 戴鹏辉 & 学号:& 22344016\\
	\hline
    日期:& 2024/5/12 & 评分: &\\
	\hline
	\end{tabularx}
\end{table}

\section{D1 \quad 锁相放大器与弱信号测量 \quad\heiti 分析与讨论}

\subsection{实验数据分析}


	\subsubsection{实验一 \quad 选择合适电流量程,设置氩管工作电压}
		
		





\end{document}
